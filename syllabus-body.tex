\section{Academic Integrity}\label{academic-integrity}

As in all courses at Vanderbilt, your work in PSCI 220 is governed by
the Honor Code. I encourage you to discuss course material and
assignments with your peers, but the written work you turn in must be
solely your own. You are required to write and sign the Honor Pledge on
all written assignments and examinations: ``I pledge on my honor that I
have neither given nor received unauthorized aid on this assignment.''

I have no tolerance for plagiarism. If you turn in plagiarized work, you
will receive a failing grade for the course and be reported to the Honor
Council. Plagiarism is not just verbatim copying and
pasting---representing someone else's ideas as your own without citing
the source is also a form of plagiarism. Ignorance of what constitutes
plagiarism is not an excuse or a defense. For more information about
what is and is not plagiarism, refer to
\href{http://www.vanderbilt.edu/student_handbook/the-honor-system/}{the
Student Handbook section on the Honor System}. Always remember:
\emph{when in doubt, cite.}

\section{Additional Concerns}\label{additional-concerns}

\subsection{Special Accommodations}\label{special-accommodations}

If you need course accommodations due to a disability, if you have
emergency medical information to share with me, or if you need special
arrangements in case the building must be evacuated, please make an
appointment with me or with the
\href{http://www.vanderbilt.edu/ead/}{Equal Opportunity, Affirmative
Action, and Disability Services Department} (2-4705) as soon as
possible.

\subsection{Classroom Recording}\label{classroom-recording}

The use of technologies for audio and video recording of lectures and
other classroom activities is allowed only with the express permission
of the instructor. In cases where recordings are allowed, such content
is restricted to personal use only unless permission is expressly
granted in writing by the instructor and by other classroom
participants, including other students. Personal use is defined as use
by an individual student for the purpose of studying or completing
course assignments. When students have permission for personal use of
recordings, they must still obtain written permission from the
instructor to share recordings with others.

For students registered with EAD and who have been approved for audio or
video recording of lectures and other classroom activities as a
reasonable accommodation, applicable federal law requires instructors to
permit those recordings. Such recordings are also limited to personal
use, except with permission of the instructor and other students in the
class.

\section{Books}\label{books}

The following books are required:

\begin{itemize}
\item
  M.S. Anderson, \href{http://amzn.com/0582212375}{\emph{The Rise of
  Modern Diplomacy, 1450--1919}} (Longman, 1993).
\item
  G.R. Berridge, \href{http://amzn.com/0230229603}{\emph{Diplomacy:
  Theory and Practice}} (Palgrave Macmillan, 2010).
\item
  Thomas Schelling, \href{http://amzn.com/0300002211}{\emph{Arms and
  Influence}} (Yale University Press, 1966).
\end{itemize}

\section{Schedule}\label{schedule}

\subsubsection{August 20--22: Diplomacy and the International
System}\label{august-2022-diplomacy-and-the-international-system}

\begin{itemize}
\itemsep1pt\parskip0pt\parsep0pt
\item
  Hedley Bull, ``Diplomacy and International Order,'' chap. 7 in
  \emph{The Anarchical Society} (Columbia University Press, 1977).
\end{itemize}

\subsubsection{August 25--29: Diplomacy in International Relations
Theory}\label{august-2529-diplomacy-in-international-relations-theory}

\emph{No class Friday, August 29.}

\begin{itemize}
\item
  Hans Morgenthau, ``Diplomacy'' and ``The Future of Diplomacy,'' chap.
  31--32 in \emph{Politics Among Nations}, 6th ed. (Alfred A. Knopf,
  1985).
\item
  Robert Keohane and Joseph S. Nye Jr., ``Realism and Complex
  Interdependence,'' chap. 2 in \emph{Power and Interdependence}, 3rd
  ed. (Longman, 2000).
\item
  Christer Jonsson and Martin Hall, ``The Study of Diplomacy,'' chap. 1
  in \emph{Essence of Diplomacy} (Palgrave Macmillan, 2005).
\item
  Robert Powell,
  \href{http://www.annualreviews.org/doi/abs/10.1146/annurev.polisci.5.092601.141138}{``Bargaining
  Theory and International Conflict,''} \emph{Annual Review of Political
  Science} 5 (2002): 1--30.
\end{itemize}

\subsubsection{September 1--5: Diplomacy through
History}\label{september-15-diplomacy-through-history}

\begin{itemize}
\item
  Anderson, \emph{The Rise of Modern Diplomacy}, entire book.
\item
  \emph{Recommended:}

  \begin{itemize}
  \itemsep1pt\parskip0pt\parsep0pt
  \item
    Garrett Mattingly, \emph{Renaissance Diplomacy} (Cosimo Classics,
    2009).
  \end{itemize}
\end{itemize}

\subsubsection{September 8--12: The Practice of
Negotiations}\label{september-812-the-practice-of-negotiations}

\begin{itemize}
\item
  Berridge, \emph{Diplomacy: Theory and Practice}, chapters 2--6.
\item
  \textbf{Add some articles or historical examples}
\end{itemize}

\subsubsection{September 15--19:
Deterrence}\label{september-1519-deterrence}

\begin{itemize}
\item
  Thomas Schelling, \emph{Arms and Influence}, chapters 1--3.
\item
  \emph{Recommended:}

  \begin{itemize}
  \itemsep1pt\parskip0pt\parsep0pt
  \item
    Daniel Ellsburg, \href{http://www.jstor.org/stable/1914513}{``The
    Crude Analysis of Strategy Choices,''} \emph{American Economic
    Review} 51, no. 2 (1961): 472--478.
  \item
    Bruce M. Russett, \href{http://www.jstor.org/stable/172796}{``The
    Calculus of Deterrence,''} \emph{Journal of Conflict Resolution} 7,
    no. 2 (1963): 97--109.
  \end{itemize}
\end{itemize}

\subsubsection{September 22--26: Spiral Models versus
Deterrence}\label{september-2226-spiral-models-versus-deterrence}

\begin{itemize}
\item
  Robert Jervis, ``Deterrence, the Spiral Model, and Intentions of the
  Adversary,'' chap. 3 in \emph{Perception and Misperception in
  International Politics} (Princeton University Press, 1976).
\item
  Robert Jervis,
  \href{http://www.jstor.org/stable/2009958}{``Cooperation under the
  Security Dilemma,''} \emph{World Politics} 30, no. 2 (1978): 167--214.
\item
  Andrew Kydd, \href{http://www.jstor.org/stable/25054007}{``Game Theory
  and the Spiral Model,''} \emph{World Politics} 49, no. 3 (1997):
  371--400.
\item
  Charles L. Glaser, \href{http://www.jstor.org/stable/25054031}{``The
  Security Dilemma Revisited,''} \emph{World Politics} 50, no. 1 (1997):
  171--201.
\end{itemize}

\subsubsection{September 29--October 3: Testing Rational Deterrence
Theory}\label{september-29october-3-testing-rational-deterrence-theory}

\begin{itemize}
\item
  Christopher H. Achen and Duncan Snidal,
  \href{http://www.jstor.org/stable/2010405}{``Rational Deterrence
  Theory and Comparative Case Studies,''} \emph{World Politics} 41, no.
  2 (1989): 143--169.
\item
  Richard Ned Lebow and Janice Gross Stein,
  \href{http://www.jstor.org/stable/2010408}{``Rational Deterrence
  Theory: I Think, Therefore I Deter,''} \emph{World Politics} 41, no. 2
  (1989): 208--224.
\item
  Richard Ned Lebow and Janice Gross Stein,
  \href{http://www.jstor.org/stable/2010415}{``Deterrence: The Elusive
  Dependent Variable,''} \emph{World Politics} 42, no. 3 (1990):
  336--369.
\item
  Paul Huth and Bruce Russett,
  \href{http://www.jstor.org/stable/2010511}{``Testing Deterrence
  Theory: Rigor Makes a Difference,''} \emph{World Politics} 42, no. 4
  (1990): 466--501.
\item
  \emph{Recommended:}

  \begin{itemize}
  \itemsep1pt\parskip0pt\parsep0pt
  \item
    James D. Fearon,
    \href{http://www.tandfonline.com/doi/abs/10.1080/03050620210390}{``Selection
    Effects and Deterrence,''} \emph{International Interactions} 28, no.
    1 (2002): 5--29.
  \end{itemize}
\end{itemize}

\subsubsection{October 6--10: Constructing
Alliances}\label{october-610-constructing-alliances}

\emph{No class Friday, October 10.}

\begin{itemize}
\item
  Stephen M. Walt, ``Explaining Alliance Formation,'' chap. 2 in
  \emph{The Origins of Alliances} (Cornell University Press, 1987).
\item
  James D. Morrow,
  \href{http://www.annualreviews.org/doi/abs/10.1146/annurev.polisci.3.1.63}{``Alliances:
  Why Write Them Down?''} \emph{Annual Review of Political Science} 3
  (2000): 63--83.
\item
  Brian Lai and Dan Reiter,
  \href{http://www.jstor.org/stable/174663}{``Democracy, Political
  Similarity, and International Alliances, 1816--1992,''} \emph{Journal
  of Conflict Resolution} 44, no. 2 (2000): 203--227.
\item
  \emph{Recommended:}

  \begin{itemize}
  \itemsep1pt\parskip0pt\parsep0pt
  \item
    Mancur Olson Jr. and Richard Zeckhauser,
    \href{http://www.jstor.org/stable/1927082}{``An Economic Theory of
    Alliances,''} \emph{Review of Economics and Statistics} 48, no. 3
    (1966): 266--279.
  \end{itemize}
\end{itemize}

\subsubsection{October 13--15: Keeping Alliance
Commitments}\label{october-1315-keeping-alliance-commitments}

\emph{No class Friday, October 17 due to fall break.}

\begin{itemize}
\item
  Brett Ashley Leeds, Andrew G. Long, and Sara McLaughlin Mitchell,
  \href{http://www.jstor.org/stable/174649}{``Reevaluating Alliance
  Reliability: Specific Threats, Specific Promises,''} \emph{Journal of
  Conflict Resolution} 44, no. 5 (2000): 686--699.
\item
  Kurt Taylor Gaubatz,
  \href{http://www.jstor.org/stable/2707000}{``Democratic States and
  Commitment in International Relations,''} \emph{International
  Organization} 50, no. 1 (1996): 109--139.
\item
  Erik Gartzke and Kristian Skrede Gleditsch,
  \href{http://www.jstor.org/stable/1519933}{``Why Democracies May
  Actually Be Less Reliable Allies,''} \emph{American Journal of
  Political Science} 48, no. 4 (2004): 775--795.
\end{itemize}

\subsubsection{October 20: Midterm Exam}\label{october-20-midterm-exam}

\subsubsection{October 22--24: Domestic Politics and
Diplomacy}\label{october-2224-domestic-politics-and-diplomacy}

\begin{itemize}
\item
  Robert D. Putnam,
  \href{http://www.jstor.org/stable/2706785}{``Diplomacy and Domestic
  Politics: The Logic of Two-Level Games,''} \emph{International
  Organization} 42, no. 3 (1988): 427--460.
\item
  James D. Fearon, \href{http://www.jstor.org/stable/2944796}{``Domestic
  Political Audiences and the Escalation of International Disputes,''}
  \emph{American Political Science Review} 88, no. 3 (1994): 577--592.
\item
  Jack Snyder and Erica D. Borghard,
  \href{http://dx.doi.org/10.1017/S000305541100027X}{``The Cost of Empty
  Threats: A Penny, Not a Pound,''} \emph{American Political Science
  Review} 105, no. 3 (2011): 437--456.
\item
  Brett Ashley Leeds,
  \href{http://www.jstor.org/stable/2991814}{``Domestic Political
  Institutions, Credible Commitments, and International Cooperation,''}
  \emph{American Journal of Political Science} 43, no. 4 (1999):
  979--1002.
\end{itemize}

\subsubsection{October 27--31: Third-Party
Mediation}\label{october-2731-third-party-mediation}

\begin{itemize}
\item
  Andrew Kydd, \href{http://www.jstor.org/stable/3186121}{``Which Side
  Are You On? Bias, Credibility, and Mediation,''} \emph{American
  Journal of Political Science} 47, no. 4 (2003): 597--611.
\item
  Katja Favretto, \href{http://www.jstor.org/stable/27798500}{``Should
  Peacemakers Take Sides? Major Power Mediation, Coercion, and Bias,''}
  \emph{American Political Science Review} 103, no. 2 (2009): 248--263.
\item
  Kyle C. Beardsley, \href{http://www.jstor.org/stable/20752196}{``Pain,
  Pressure, and Political Cover: Explaining Mediation Incidence,''}
  \emph{Journal of Peace Research} 47, no. 4 (2010): 395--406.
\item
  Shawn Ling Ramirez,
  \href{http://slramirez.github.io/lionsden/med20130630.pdf}{``Diplomatic
  Flexibility in the Shadow of an Audience: The Double-Edged Sword of
  Private Mediation''} (working paper, Emory University, 2013).
\end{itemize}

\subsubsection{November 3--7:
Peacekeeping}\label{november-37-peacekeeping}

\begin{itemize}
\item
  Virginia Page Fortna,
  \href{http://www.jstor.org/stable/3594855}{``Scraps of Paper?
  Agreements and the Durability of Peace,''} \emph{International
  Organization} 57, no. 2 (2003): 337--372.
\item
  Suzanne Werner and Amy Yuen,
  \href{http://www.jstor.org/stable/3877905}{``Making and Keeping
  Peace,''} \emph{International Organization} 59, no. 2 (2005):
  261--292.
\item
  J. Michael Greig and Paul F. Diehl,
  \href{http://www.jstor.org/stable/3693503}{``The
  Peacekeeping--Peacemaking Dilemma,''} \emph{International Studies
  Quarterly} 49, no. 4 (2005): 621--645.
\end{itemize}

\subsubsection{November 10--14: Diplomacy through International
Organizations}\label{november-1014-diplomacy-through-international-organizations}

\begin{itemize}
\item
  Lisa Martin, \href{http://www.jstor.org/stable/2706874}{``Interests,
  Power, and Multilateralism,''} \emph{International Organization} 46,
  no. 4 (1992): 765--792.
\item
  Jennifer Mitzen, \href{http://www.jstor.org/stable/30038948}{``Reading
  Habermas in Anarchy: Multilateral Diplomacy and Global Public
  Spheres,''} \emph{American Political Science Review} 99, no. 3 (2005):
  401--417.
\item
  Alexander Thompson,
  \href{http://www.jstor.org/stable/3877866}{``Coercion through IOs: The
  Security Council and the Logic of Information Transmission,''}
  \emph{International Organization} 60, no. 1 (2006): 1--34.
\item
  Jon Pevehouse and Bruce Russett,
  \href{http://www.jstor.org/stable/3877853}{``Democratic International
  Governmental Organizations Promote Peace,''} \emph{International
  Organization} 60, no. 4 (2006): 969--1000.
\end{itemize}

\subsubsection{November 17--21: Negotiating with Non-State
Actors}\label{november-1721-negotiating-with-non-state-actors}

\begin{itemize}
\item
  Peter C. Sederberg,
  \href{http://www.jstor.org/stable/425666}{``Conciliation as
  Counter-Terrorist Strategy,''} \emph{Journal of Peace Research} 32,
  no. 3 (1995): 295--312.
\item
  Barbara F. Walter, \href{http://www.jstor.org/stable/2703607}{``The
  Critical Barrier to Civil War Settlement,''} \emph{International
  Organization} 51, no. 3 (1997): 335--364.
\item
  Navin A. Bapat, \href{http://www.jstor.org/stable/3693558}{``State
  Bargaining with Transnational Terrorist Groups,''} \emph{International
  Studies Quarterly} 50, no. 1 (2006): 213--230.
\end{itemize}

\subsubsection{December 1--3: Student
Presentations}\label{december-13-student-presentations}
